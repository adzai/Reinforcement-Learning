\documentclass[lettersize,journal]{IEEEtran}
\usepackage{amsmath,amsfonts}
\usepackage{algorithmic}
\usepackage{array}
\usepackage[caption=false,font=normalsize,labelfont=sf,textfont=sf]{subfig}
\usepackage{textcomp}
\usepackage{stfloats}
\usepackage{url}
\usepackage{verbatim}
\usepackage{graphicx}
% \hyphenation{op-tical net-works semi-conduc-tor IEEE-Xplore}
\def\BibTeX{{\rm B\kern-.05em{\sc i\kern-.025em b}\kern-.08em
    T\kern-.1667em\lower.7ex\hbox{E}\kern-.125emX}}
\usepackage{balance}
\begin{document}
\title{Q-Learning in the snake game}
\author{Adam Zaiter}

\maketitle

\begin{abstract}
\end{abstract}

\begin{IEEEkeywords}
\end{IEEEkeywords}


\section{Introduction}
This paper will attempt to implement the Q-learning algorithm
for a basic game of snake. A hypothesis about the expected learning
outcome will be established and reviewed after implementing the
Q-Learing algorithm. The implementation will be done in Python
and a custom snake game environment will be created.
This environment will allow for training of the agent with different
hyperparameters as well as visualizing a game played by the
trained agent.


Basic knowledge of reinforcement learning is assumed, as this paper
is implementing concepts from the Szepesvari paper.

\section{Snake game}
The snake game is programmed in python.
It supports square grids of any size that is $>= 3$.
The snake starts in the middle of the grid with a head of
1 square in size and attached
body with 2 squares in size. There also is food spawned in a random
square that is not occupied by the snake. The objective for the
snake is to eat as much food as possible. If the food is eaten,
the snake gains + 10 score and grows an additional body part of
1 square in size. If the snake collides with the borders of the grid
or the head collides with it's own body, the game ends.\label{game-over}

\section{Episode}
An episode contains all of the actions taken in one game
of the snake. The episode ends when a terminal state is
reached (game ends), described in section \ref{game-over}. In the case of the snake game that is either
the head of the snake colliding with a border or the head
colliding with it's own body.

\section{Step}
A step is the performance of one action in a given state.



\section{Q-Learning}
Q-Learning finds the optimal policy by learning
the best Q values for each state-action pair.

The Q function accepts the current state and an action. It then
returns the expected reward for taking said action for the given
state.

The agent then plays episodes. For each step in the episode
the Q values in the Q table are updated using the Bellman
equation. This process goes on until the q function converges
to the optimal function q*.

\section{Hypothesis}
The goal is to train an agent that will play the snake game
with any chosen grid size as best as possible using just Q-Learning.
Given the attributes of Q-Learning, it should be possible to train
an agent, that plays the game of snake at a very good level. It
is not expected to be perfect, but it should play as close to it
as possible.


\section{Environment}
The environment mimics functionality provided by environments in
OpenAI gyms. This made the process of applying the Q-Learning
algorithm quite simple, as the OpenAI environments are designed
with reinforcement learning in mind.

The environment provides three actions for the agent to take:
\begin{enumerate}
    \item Turn the snake head left.
    \item Turn the snake head right.
    \item Continue heading straight.
\end{enumerate}
These values are encoded as integers 0, 1, 2 and aliased as
constants LEFT, RIGHT and STRAIGHT in the code.

The environment does not provide the full state of the board,
as there are simply too many states to represent for a grid
of variable size. Thus, the environment is restricted to
this basic information:
\begin{itemize}
    \item Information about danger. It indicates if after taking an action from the action space in the current state
        the game will be over. This can be expanded for a lookahead of one or more turns in the future.
    \item Information about food position. The values are left, right, up and down.
    \item Information about the snake's current direction. There are four values: left, right, up and down.
\end{itemize}

After gathering this information, it is encoded in a 2D array. For example, the array for food position
is an array of four values. Each position of the array corresponds to a value (left, right, up, down).
Thus if the food is left, the array on position left will have the value of 1 (true). Otherwise, the
value will be 0 (false). This applies for all positions. In the end, three individual arrays for danger,
food and direction are created and are returned concatenated as the observation array.

\section{Observations}
The snake seems to learn better in smaller environments.

Even with limited information the snake learns the strategy
of creating as much space as possible for it's own body.
This results in the snake hugging the borders as much
as possible.
Thus, it takes longer paths to get to the reward (food) which
might seem strange early in the game.
However when the snake's body is longer after collecting a
few rewards, the benefits of this strategy are fully visible.


Since the environment is custom, some thought had to be given
to what kind of rewards the agent should get as well as their
quantity.


% It is
% probably not feasible to achieve superhuman performance however,
% because of the limited computing resources. The state space
% is simply too large when taking into account games of snake
% with any given grid size.



\section{State space}
For this project, the computing resources are somewhat limited and
as such, these limited resources have to be taken into consideration
when designing the state and action spaces.
The state space for a snake game with a variable size would be
too big, if it were taken as is. That is, because the snake and
it's body adds many additional states, as the snake can grow it's body to
be the same size as all of the squares in the grid. Also, the head is distinct
from the other body parts and would have to be accounted for in all of the states.


\section{Training}
Best results were seen when training on a 3x3 grid.
The training did not take as much time, as fewer moves
were made on average.

% \begin{thebibliography}{1}
% \bibitem{Startupsum}
    % Startupsum - Lorem Ipsum for Startups. 2021. Startupsum - Lorem Ipsum for startups. A perfect blend of being professional and having fresh filler content.. [online] Available at: \url{https://startupsum.com/} [Accessed 1 December 2021].

% \end{thebibliography}

\end{document}
